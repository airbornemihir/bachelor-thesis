\documentclass{beamer}
\usetheme{default}
\usepackage[algo2e]{algorithm2e}
\usepackage{subfig}

\title{Tools and Algorithms for Deciding Relations on Timed Automata}
\subtitle{B Tech project, supervised by S Arun-Kumar, verification group}
\author{Mihir Mehta}
\institute[IITD]{
  Department of Computer Science and Engineering\\
  Indian Institute of Technology, Delhi\\[1ex]
  \texttt{cs1090197@cse.iitd.ac.in}
}
\date[May 2013]{May 12, 2013}

\begin{document}

%--- the titlepage frame -------------------------%
\begin{frame}[plain]
  \titlepage
\end{frame}

\AtBeginSection[]{

  \frame<beamer>{ 

    \frametitle{Outline}   

    \tableofcontents[currentsection,currentsubsection] 

  }

}

\section{Automata without timing and relations on them}

\begin{frame}{Labeled transition systems}
  \begin{definition}
    \emph{Labelled Transition System}: A labelled transition system (LTS)
    \cite{Keller:1976:FVP:360248.360251} is an automaton which is
    described by
    \begin{itemize}
    \item $S$, a set of \emph{states} 
    \item $Act$, a set of \emph{actions}
    \item $\rightarrow \subseteq S \times Act \times S$, a \emph{transition
      relation}.
    \item optionally, $I \subseteq S$ ,a set of initial states. If there
      is exactly one initial state, then the LTS is said to be \emph{rooted}.
    \end{itemize}
  \end{definition}
\end{frame}

\begin{frame}[allowframebreaks]{Relations on LTS}

  \begin{definition}
    \emph{Strong bisimulation}:A binary relation $R$ on the states of an
    LTS is a strong bisimulation if and only if, for all
    $(s_1, s_2)$ $\epsilon$ $R$ and $a$ $\epsilon$ $Act .$\\
    $\forall s_1' (s_1 \xrightarrow{a} s_1' \Rightarrow \exists s_2'
    . (s_2 \xrightarrow{a} s_2' \wedge (s_1', s_2')$ $\epsilon$ $R ) )
    \wedge $ \\
    $\forall s_2' (s_2 \xrightarrow{a} s_2' \Rightarrow \exists s_1'
    . (s_1 \xrightarrow{a} s_1' \wedge (s_1', s_2')$ $\epsilon$ $R ) )$
  \end{definition}

  \begin{definition}
    It can be shown that the union of
    all strong bisimulations over the set of states is a strong
    bisimulation. This binary relation is called \emph{strong
      bisimilarity}, denoted by $\sim$.
  \end{definition}

\end{frame}

\section{Timed automata and relations on them}

\begin{frame}{Timed Automata}
  \begin{definition}
    \emph{Timed Automaton}: A timed automaton
    \cite{Alur94atheory} over a finite set of clocks $C$
    and a finite set of actions $Act$ is a 4-tuple $(L, l_{0}, E, I)$.
    \begin{itemize}
    \item $L$ is a finite set of locations.
    \item $l_{0}$ is the initial location.
    \item $E \subseteq L \times B(C) \times Act \times 2^{C} \times L$
      is a finite set of edges.
    \item $I: L \rightarrow B(C)$ assigns invariants to each edge
      location.
    \item $B(C)$ is the set of clock constraints over C.
    \end{itemize}
  \end{definition}
\end{frame}

\begin{frame}[allowframebreaks]{Relations on timed automata}
\begin{definition} 
\label{def:stab} 
  \emph{Strong time abstracted bisimulation}: A binary relation
  $R$ is a strong time abstracted bisimulation (STaB) if and only if, for all
  $(s_1, s_2)$ $\epsilon$ $R$ , $a$ $\epsilon$ $Act $, $d$ $\epsilon$ $R_{\ge 0}$\\
  $\forall s_1' (s_1 \xrightarrow{a} s_1' \Rightarrow \exists s_2'
  . (s_2 \xrightarrow{a} s_2' \wedge (s_1', s_2')$ $\epsilon$ $R ) )
  \wedge $ \\
  $\forall s_2' (s_2 \xrightarrow{a} s_2' \Rightarrow \exists s_1'
  . (s_1 \xrightarrow{a} s_1' \wedge (s_1', s_2')$ $\epsilon$ $R ) ) \wedge $ \\
  $\forall s_1' (s_1 \xrightarrow{d} s_1' \Rightarrow \exists (s_2',
  d')
  . (s_2 \xrightarrow{d'} s_2' \wedge (s_1', s_2')$ $\epsilon$ $R ) )
  \wedge $ \\
  $\forall s_2' (s_2 \xrightarrow{d} s_2' \Rightarrow \exists (s_1', d')
  . (s_1 \xrightarrow{d'} s_1' \wedge (s_1', s_2')$ $\epsilon$ $R ) ) $ \\
\end{definition}

It can be shown that the union of all strong time abstracted
  bisimulations over the set of (location, valuation) pairs is a
  strong time abstracted bisimulation. This binary relation is called
  \textit{strong time abstracted bisimilarity}.

\begin{definition}
  \emph{Time abstracted delay bisimulation}: A binary relation
  $R$ is a time abstracted delay bisimulation (TadB) if and only if, for all
  $(s_1, s_2)$ $\epsilon$ $R$ , $a$ $\epsilon$ $Act $, $d$ $\epsilon$ $R_{\ge 0}$\\
  $\forall s_1' (s_1 \xrightarrow{a} s_1' \Rightarrow \exists (s_2', d)
  . (s_2 \xrightarrow{d} \xrightarrow{a} s_2' \wedge (s_1', s_2')$ $\epsilon$ $R ) )
  \wedge $ \\
  $\forall s_2' (s_2 \xrightarrow{a} s_2' \Rightarrow \exists (s_1', d)
  . (s_1 \xrightarrow{d} \xrightarrow{a} s_1' \wedge (s_1', s_2')$
  $\epsilon$ $R ) ) 
  \wedge $ \\
  $\forall s_1' (s_1 \xrightarrow{d} s_1' \Rightarrow \exists (s_2',
  d')
  . (s_2 \xrightarrow{d'} s_2' \wedge (s_1', s_2')$ $\epsilon$ $R ) )
  \wedge $ \\
  $\forall s_2' (s_2 \xrightarrow{d} s_2' \Rightarrow \exists (s_1', d')
  . (s_1 \xrightarrow{d'} s_1' \wedge (s_1', s_2')$ $\epsilon$ $R ) ) $ \\
\end{definition}

It can be shown that the
  union of all time abstracted delay bisimulations over the set of
  (location, valuation) pairs is a time abstracted delay
  bisimulation. This binary relation is called \textit{time abstracted
    delay bisimilarity}.

\begin{definition}
  \emph{Time abstracted observational bisimulation}: A binary relation
  $R$ is a time abstracted observational bisimulation (TaoB) if and only if, for all
  $(s_1, s_2)$ $\epsilon$ $R$ , $a$ $\epsilon$ $Act $, $d$ $\epsilon$ $R_{\ge 0}$\\
  $\forall s_1' (s_1 \xrightarrow{a} s_1' \Rightarrow \exists (s_2',
  d, d') . (s_2 \xrightarrow{d} \xrightarrow{a} \xrightarrow{d'} s_2'
  \wedge (s_1', s_2')$ $\epsilon$ $R ) ) \wedge $ \\
  $\forall s_2' (s_2 \xrightarrow{a} s_2' \Rightarrow \exists (s_1',
  d, d') . (s_1 \xrightarrow{d} \xrightarrow{a} \xrightarrow{d'} s_1'
  \wedge (s_1', s_2')$ $\epsilon$ $R ) ) \wedge $ \\
  $\forall s_1' (s_1 \xrightarrow{d} s_1' \Rightarrow \exists (s_2',
  d')
  . (s_2 \xrightarrow{d'} s_2' \wedge (s_1', s_2')$ $\epsilon$ $R ) )
  \wedge $ \\
  $\forall s_2' (s_2 \xrightarrow{d} s_2' \Rightarrow \exists (s_1', d')
  . (s_1 \xrightarrow{d'} s_1' \wedge (s_1', s_2')$ $\epsilon$ $R ) ) $ \\
\end{definition}

It can be shown that
  the union of all time abstracted observational bisimulations over the
  set of (location, valuation) pairs is a time abstracted observational
  bisimulation. This binary relation is called \textit{time abstracted
    observational bisimilarity}.

\end{frame}

\section{Algorithms}

\begin{frame}[fragile]
  \frametitle{Creating the zone valuation graph}

  \begin{algorithm2e}[H]
    Initialise the queue $Q$ with a single element $(null, null, l_0)$\;
    Initialise the graph $zone\_graph$ with a single node $(l_0, v_0 \uparrow)$
    with an $\epsilon$ self-loop\;
    \While{$Q$ is not empty}{
      Dequeue $(l_{parent}, t, l_{child})$ from $Q$\;
      \If{$l_{parent} \neq null$}{
        \ForEach{zone $Z_{parent}$ of $l_{parent}$}{
          Add new zones to the zones of $l_{child}$ so that all zones
          reachable from $Z_{parent}$ are represented\;
          Abstract if necessary\;
          Update edges from $Z_{parent}$ to the new zones of $l_{child}$
          \If{new zones are created in $l_{child}$ or $l_{parent}$ is null}{
            \ForEach{outgoing transition $t'$ of $l_{child}$}{
              Enqueue $(l_{child}, t', \texttt{t'.target})$ in $Q$\;
            }
          }
        }
      }
      Set $new\_zone$\;
      \While{$new\_zone$}{
        Reset $new\_zone$\;
        \ForEach{transition $t$ in the timed automaton}{
          Split the zones of \texttt{t.source} to be stable with respect to the
          zones of \texttt{t.target}\;
          Update edges accordingly\;
          \If{new zones are created in \texttt{t.source}}{
            Set $new\_zone$\;
          }
        }
      }
    }
    Generate $zone\_valuation\_graph$ by applying Fernandez' algorithm to $zone\_graph$\;
    Return $zone\_graph$\;
  \end{algorithm2e}

\end{frame}

\begin{frame}{Zone valuation graph example}
  \begin{figure}
    \centering
    \def\svgwidth{0.6\columnwidth}
    \input{breaking2.pdf_tex}
    \caption{Timed automaton. Here, the states are \{0\}, the actions
      are \{a\}, and the clocks are \{X, Y\}.}
  \end{figure}
\end{frame}

\begin{frame}{Zone valuation graph example}
  \begin{figure}
    \centering
    \def\svgwidth{0.4\columnwidth}
    \input{breaking2-zones01.pdf_tex}
    \caption{Zones after one iteration.}
  \end{figure}
\end{frame}

\begin{frame}[shrink=20]{Zone valuation graph example}
  \begin{figure}
    \centering
    \def\svgwidth{0.7\columnwidth}
    \input{breaking2-zones02.pdf_tex}
    \caption{Zones after two iterations.}
  \end{figure}
\end{frame}

\begin{frame}[shrink=20]{Zone valuation graph example}
  \begin{figure}
    \centering
    \def\svgwidth{0.9\columnwidth}
    \input{breaking2-zones03.pdf_tex}
    \caption{Zones after three iterations without abstraction.}
  \end{figure}
\end{frame}

\begin{frame}[shrink=20]{Zone valuation graph example}
  \begin{figure}
    \centering
    \def\svgwidth{1.2\columnwidth}
    \input{breaking2-zones-abstracted.pdf_tex}
    \caption{Zones after three iterations with abstraction.}
  \end{figure}
\end{frame}

\begin{frame}[shrink=60]{Zone valuation graph example}
  \begin{figure}
    \centering
    \def\svgwidth{2.7\columnwidth}
    \input{breaking2-zones-quotient.pdf_tex}
    \caption{Zone graph with bisimilarity classes.}
  \end{figure}
\end{frame}

\begin{frame}{Verifying relations on pairs of timed automata.}
  \begin{itemize}
  \item Origin for this algorithm - \cite{arun2006bisimilarities} 
  \item General method to compute $(\rho, \sigma)$-bisimilarities on
    two LTS, starting from their initial locations.
  \item Can be adapted for a certain class of timed and time
    abstracted relations by using zone valuation graphs.
  \item For every relation $R$ satisfying this property, functions
    $f_P$ and $f_Q$ must exist such that the proposition $s_P R s_Q$
    resolves to one of these:
    \begin{itemize}
    \item yes
    \item no
    \item if and only if 
      \begin{align*} 
        &\forall (s_P', L_Q') \in f_P(s_P): \exists s_Q' \in
        L_Q': s_P' R s_Q' \quad \wedge \\
        &\forall (L_P', s_Q') \in f_Q(s_Q): \exists s_P' \in
        L_P': s_P' R s_Q'
      \end{align*} 
    \end{itemize}
  \end{itemize}
\end{frame}

\begin{frame}[shrink=20]{Verifying relations on pairs of timed automata.}
  \begin{itemize}
  \item For STaB, we define $f_P$ and $f_Q$ as
    \begin{align*}
      f_P(s_P) = & \{(s_P', L_Q') | s_P \xrightarrow{a} s_P', 
      L_Q=\{ s_Q' | s_Q \xrightarrow{a} s_Q'\}\} \\
      \cup & \{(s_P', L_Q') | s_P \xrightarrow{\epsilon} s_P', 
      L_Q=\{ s_Q' | s_Q \xrightarrow{\epsilon} s_Q'\}\} \\
      f_Q(s_Q) = & \{(L_P', s_Q') | s_Q \xrightarrow{a} s_Q', 
      L_P=\{ s_P' | s_P \xrightarrow{a} s_P'\}\} \\
      \cup & \{(L_P', s_Q') | s_Q \xrightarrow{\epsilon} s_Q', 
      L_P=\{ s_P' | s_P \xrightarrow{\epsilon} s_P'\}\} 
    \end{align*}

  \item For TadB, we define $f_P$ and $f_Q$ as
    \begin{align*}
      f_P(s_P) = & \{(s_P', L_Q') | s_P \xrightarrow{a} s_P', 
      L_Q=\{ s_Q' | s_Q \xrightarrow{\epsilon}\xrightarrow{a} s_Q'\}\} \\
      \cup & \{(s_P', L_Q') | s_P \xrightarrow{\epsilon} s_P', 
      L_Q=\{ s_Q' | s_Q \xrightarrow{\epsilon} s_Q'\}\} \\
      f_Q(s_Q) = & \{(L_P', s_Q') | s_Q \xrightarrow{a} s_Q', 
      L_P=\{ s_P' | s_P \xrightarrow{\epsilon}\xrightarrow{a} s_P'\}\} \\
      \cup & \{(L_P', s_Q') | s_Q \xrightarrow{\epsilon} s_Q', 
      L_P=\{ s_P' | s_P \xrightarrow{\epsilon} s_P'\}\} 
    \end{align*}

  \item For TaoB, we define $f_P$ and $f_Q$ as
    \begin{align*}
      f_P(s_P) = & \{(s_P', L_Q') | s_P \xrightarrow{a} s_P', 
      L_Q=\{ s_Q' | s_Q \xrightarrow{\epsilon}\xrightarrow{a}\xrightarrow{\epsilon} s_Q'\}\} \\
      \cup & \{(s_P', L_Q') | s_P \xrightarrow{\epsilon} s_P', 
      L_Q=\{ s_Q' | s_Q \xrightarrow{\epsilon} s_Q'\}\} \\
      f_Q(s_Q) = & \{(L_P', s_Q') | s_Q \xrightarrow{a} s_Q', 
      L_P=\{ s_P' | s_P \xrightarrow{\epsilon}\xrightarrow{a}\xrightarrow{\epsilon} s_P'\}\} \\
      \cup & \{(L_P', s_Q') | s_Q \xrightarrow{\epsilon} s_Q', 
      L_P=\{ s_P' | s_P \xrightarrow{\epsilon} s_P'\}\} 
    \end{align*}

  \end{itemize}
\end{frame}

\begin{frame}[fragile, shrink=60]
  \frametitle{Verifying relations on pairs of timed automata.}
\begin{procedure}[H]
  \caption{CheckStatesRelation($P$, $Q$, $s_{P}$, $s_{Q}$,
    $yes\_ table$, $no\_ table$)}
  \label{algorithm:checkstatesrelation}
  \SetKwFunction{lookup}{lookup}
  \SetKwFunction{insert}{insert}
  \SetKwFunction{remove}{remove}
  \Begin{
      \eIf{\lookup{$yes\_ table$, $s_{P}$, $s_{Q}$}}{
        \KwRet true\;
      }{
        \eIf{\lookup{$yes\_ table$, $s_{P}$, $s_{Q}$}}{
          \KwRet false\;
        }{
          \insert{$yes\_ table$, $s_{P}$, $s_{Q}$}\;
          Set $v_P$\;
          \ForEach{$(s_P', L_Q')$ in $f_P(s_P)$}{
            Reset $v_P$\;
            \ForEach{$s_Q'$ in $L_Q'$}{
              \If{\CheckStatesRelation{$P$, $Q$, $s_{P}'$, $s_{Q}'$,
                  $yes\_ table$, $no\_ table$}}{
                Set $v_P$\;
              }
            }
          }
          Set $v_Q$\;
          \ForEach{$(s_Q', L_P')$ in $f_Q(s_Q)$}{
            Reset $v_Q$\;
            \ForEach{$s_P'$ in $L_P'$}{
              \If{\CheckStatesRelation{$P$, $Q$, $s_{P}'$, $s_{Q}'$,
                  $yes\_ table$, $no\_ table$}}{
                Set $v_Q$\;
              }
            }
          }
          \eIf{$v_P$ $\wedge$ $v_Q$}{
            \KwRet true\;
          }{
            \remove{$yes\_ table$, $s_{P}$, $s_{Q}$}\;
            \insert{$yes\_ table$, $s_{P}$, $s_{Q}$}\;
            \KwRet false\;
          }
        }
      }
    }

\end{procedure}

\end{frame}

\begin{frame}[fragile]
  \frametitle{Verifying relations on pairs of timed automata.}

\begin{procedure}[H]
  \caption{CheckAutomataRelation($T_P$, $T_Q$)}
  \label{algorithm:checkautomatarelation}
  \SetKwFunction{CheckStatesRelation}{CheckStatesRelation}
  \Begin{
      Create zone valuation graphs $G_P$, $G_Q$ of $T_P$, $T_Q$\;
      Find the zone $s_P$ in $G_P$ which contains the initial
      state of $T_P$\;
      Find the zone $s_Q$ in $G_Q$ which contains the initial
      state of $T_Q$\;
      Initialise $yes\_ table$ and $no\_ table$ to empty tables\;
      \KwRet \CheckStatesRelation{$G_P$, $G_Q$, $s_P$, $s_Q$,
        $yes\_ table$, $no\_ table$}\;
    }
\end{procedure}

\end{frame}

\begin{frame}[shrink=20]{Verifying relations: example}

\begin{figure}%
\centering
\subfloat[First]{
    \def\svgwidth{0.6\columnwidth}
    \input{pair03first.pdf_tex}
}\qquad
\subfloat[Second]{
    \def\svgwidth{0.6\columnwidth}
    \input{pair03second.pdf_tex}
}\\
\caption{Timed automata.}
\label{pair03}
\end{figure}

\end{frame}

\begin{frame}[shrink=40]{Verifying relations: example}

\begin{figure}%
\centering
\subfloat[First]{
    \def\svgwidth{0.8\columnwidth}
    \input{pair03first-quotient.pdf_tex}
}\qquad
\subfloat[Second]{
    \def\svgwidth{0.8\columnwidth}
    \input{pair03second-quotient.pdf_tex}
}\\
\caption{Zone valuation graphs.}
\label{pair03}
\end{figure}

\end{frame}

\begin{frame}[allowframebreaks]
  \frametitle{References}
  \bibliographystyle{splncs}
  \bibliography{presentation}
\end{frame}

\end{document}
