\documentclass{beamer}
\usetheme{default}
\usepackage[algo2e]{algorithm2e}
\usepackage{subfig}

\title{Tools and Algorithms for Deciding Relations on Timed Automata}
\subtitle{B Tech project, supervised by S. Arun-Kumar, verification group}
\author{Mihir Mehta}
\institute[IITD]{
  Department of Computer Science and Engineering\\
  Indian Institute of Technology, Delhi\\[1ex]
  \texttt{cs1090197@cse.iitd.ac.in}
}
\date[May 2013]{May 13, 2013}

\begin{document}

%--- the titlepage frame -------------------------%
\begin{frame}[plain]
  \titlepage
\end{frame}

\AtBeginSection[]{

  \frame<beamer>{ 

    \frametitle{Outline}   

    \tableofcontents[currentsection,currentsubsection] 

  }

}

\section{Automata without timing and relations on them}

\begin{frame}{Labeled transition systems}
  \begin{definition}
    \emph{Labelled Transition System}: A labelled transition system (LTS)
    \cite{Keller:1976:FVP:360248.360251} is an automaton which is
    described by
    \begin{itemize}
    \item $S$, a set of \emph{states} 
    \item $Act$, a set of \emph{actions}
    \item $\rightarrow \subseteq S \times Act \times S$, a \emph{transition
      relation}.
    \item optionally, $I \subseteq S$ ,a set of initial states. If there
      is exactly one initial state, then the LTS is said to be \emph{rooted}.
    \end{itemize}
  \end{definition}
\end{frame}

\begin{frame}[allowframebreaks]{Relations on LTS}

  \begin{definition}
    \emph{Strong bisimulation}:A binary relation $R$ on the states of an
    LTS is a strong bisimulation if and only if, for all
    $(s_1, s_2)$ $\epsilon$ $R$ and $a$ $\epsilon$ $Act .$\\
    $\forall s_1' (s_1 \xrightarrow{a} s_1' \Rightarrow \exists s_2'
    . (s_2 \xrightarrow{a} s_2' \wedge (s_1', s_2') \in R ) )
    \wedge $ \\
    $\forall s_2' (s_2 \xrightarrow{a} s_2' \Rightarrow \exists s_1'
    . (s_1 \xrightarrow{a} s_1' \wedge (s_1', s_2') \in R ) )$
  \end{definition}

  \begin{definition}
    It can be shown that the union of
    all strong bisimulations over the set of states is a strong
    bisimulation. This binary relation is called \emph{strong
      bisimilarity}, denoted by $\sim$.
  \end{definition}

\end{frame}

\section{Timed automata and relations on them}

\begin{frame}{Timed Automata}
  \begin{definition}
    \emph{Timed Automaton}: A timed automaton
    \cite{Alur94atheory} over a finite set of clocks $C$
    and a finite set of actions $Act$ is a 4-tuple $(L, l_{0}, E, I)$.
    \begin{itemize}
    \item $L$ is a finite set of locations.
    \item $l_{0}$ is the initial location.
    \item $E \subseteq L \times B(C) \times Act \times 2^{C} \times L$
      is a finite set of edges.
    \item $I: L \rightarrow B(C)$ assigns invariants to each edge
      location.
    \item $B(C)$ is the set of clock constraints over C.
    \end{itemize}
  \end{definition}
\end{frame}

\begin{frame}{Relations on timed automata}
\begin{definition}  \label{def:stab} \emph{Strong time abstracted bisimulation}: A binary relation
  $R$ over the states of a timed automaton is a strong time abstracted
  bisimulation (STaB) if and only if, for all
  $(s_1, s_2) \in R$ , $a \in Act$, $d \in R_{\ge 0}$\\
  $\forall s_1' (s_1 \xrightarrow{a} s_1' \Rightarrow \exists s_2'
  . (s_2 \xrightarrow{a} s_2' \wedge (s_1', s_2') \in R ) )
  \wedge $ \\
  $\forall s_2' (s_2 \xrightarrow{a} s_2' \Rightarrow \exists s_1'
  . (s_1 \xrightarrow{a} s_1' \wedge (s_1', s_2') \in R ) ) \wedge $ \\
  $\forall s_1' (s_1 \xrightarrow{d} s_1' \Rightarrow \exists (s_2',
  d')
  . (s_2 \xrightarrow{d'} s_2' \wedge (s_1', s_2') \in R ) )
  \wedge $ \\
  $\forall s_2' (s_2 \xrightarrow{d} s_2' \Rightarrow \exists (s_1', d')
  . (s_1 \xrightarrow{d'} s_1' \wedge (s_1', s_2') \in R ) ) $ \\
\end{definition}

It can be shown that the union of all strong time abstracted
  bisimulations over the set of (location, valuation) pairs is a
  strong time abstracted bisimulation. This binary relation is called
  \textit{strong time abstracted bisimilarity}.
\end{frame}

\begin{frame}{Relations on timed automata}
\begin{definition} \label{def:tadb} \emph{Time abstracted delay bisimulation}: A binary relation
  $R$ over the states of a timed automaton is a time abstracted delay
  bisimulation (TadB) if and only if, for all
  $(s_1, s_2) \in R$ , $a \in Act$, $d \in R_{\ge 0}$\\
  $\forall s_1' (s_1 \xrightarrow{a} s_1' \Rightarrow \exists (s_2', d)
  . (s_2 \xrightarrow{d} \xrightarrow{a} s_2' \wedge (s_1', s_2') \in R ) )
  \wedge $ \\
  $\forall s_2' (s_2 \xrightarrow{a} s_2' \Rightarrow \exists (s_1', d)
  . (s_1 \xrightarrow{d} \xrightarrow{a} s_1' \wedge (s_1', s_2')$
  $\epsilon$ $R ) ) 
  \wedge $ \\
  $\forall s_1' (s_1 \xrightarrow{d} s_1' \Rightarrow \exists (s_2',
  d')
  . (s_2 \xrightarrow{d'} s_2' \wedge (s_1', s_2') \in R ) )
  \wedge $ \\
  $\forall s_2' (s_2 \xrightarrow{d} s_2' \Rightarrow \exists (s_1', d')
  . (s_1 \xrightarrow{d'} s_1' \wedge (s_1', s_2') \in R ) ) $ \\
\end{definition}

It can be shown that the
  union of all time abstracted delay bisimulations over the set of
  (location, valuation) pairs is a time abstracted delay
  bisimulation. This binary relation is called \textit{time abstracted
    delay bisimilarity}.
\end{frame}

\begin{frame}{Relations on timed automata}
\begin{definition} \label{def:taob} \emph{Time abstracted observational bisimulation}: A binary relation
  $R$ over the states of a timed automaton is a time abstracted
  observational bisimulation (TaoB) if and only if, for all
  $(s_1, s_2) \in R$ , $a \in Act$, $d \in R_{\ge 0}$\\
  $\forall s_1' (s_1 \xrightarrow{a} s_1' \Rightarrow \exists (s_2',
  d, d') . (s_2 \xrightarrow{d} \xrightarrow{a} \xrightarrow{d'} s_2'
  \wedge (s_1', s_2') \in R ) ) \wedge $ \\
  $\forall s_2' (s_2 \xrightarrow{a} s_2' \Rightarrow \exists (s_1',
  d, d') . (s_1 \xrightarrow{d} \xrightarrow{a} \xrightarrow{d'} s_1'
  \wedge (s_1', s_2') \in R ) ) \wedge $ \\
  $\forall s_1' (s_1 \xrightarrow{d} s_1' \Rightarrow \exists (s_2',
  d')
  . (s_2 \xrightarrow{d'} s_2' \wedge (s_1', s_2') \in R ) )
  \wedge $ \\
  $\forall s_2' (s_2 \xrightarrow{d} s_2' \Rightarrow \exists (s_1', d')
  . (s_1 \xrightarrow{d'} s_1' \wedge (s_1', s_2') \in R ) ) $ \\
\end{definition}

It can be shown that
  the union of all time abstracted observational bisimulations over the
  set of (location, valuation) pairs is a time abstracted observational
  bisimulation. This binary relation is called \textit{time abstracted
    observational bisimilarity}.

\end{frame}

\begin{frame}{Difference bound matrices}
\begin{definition}
  \emph{Difference bound matrix}: A difference bound matrix (DBM) is a
  representation of a convex polyhedron on a set of clocks $\{x_{1},
  \ldots x_{n}\}$ in the form of an $(n+1) \times (n+1)$
  matrix $M$, each element of which takes the form $(m_{ij}, \smile
  _{ij})$, where $m_{ij}$ is an integer and $\smile_{ij}$ $\epsilon$
  $\{ <, \leq\}$. Assuming $x_0$ to be a clock always valued at zero,
  the associated polyhedron is given by
  \begin{displaymath}
    \bigcap_{0 \leq i, j \leq n}(x_{i} - x_{j} \smile_{ij} m_{ij})
  \end{displaymath}
\end{definition}
\end{frame}

\begin{frame}{Abstractions}
Abstractions serve to cap the number of zones in a
zone graph by reducing zones to equivalent regions which contain them,
thus ensuring termination of zone creation algorithms which use
forward analysis.

In this implementation we use a simplified version of the
\emph{maximum constants} abstraction described in
\cite{Behrmann03staticguard}.

Algebraically, our abstraction can be thus described: given a set of
clocks $\{x_i | 1 \leq i \leq n \}$, a maximum constant $k$ over all
clocks, and a DBM $M = \langle (m_{ij}, \smile _{ij})\rangle _{0 \leq
  i,j \leq n} $, we can replace $M$ with $M' = \langle m'_{ij}, \smile
'_{ij}\rangle _{0 \leq i,j \leq n} $ where
\begin{displaymath}
  (m'_{ij}, \smile'_{ij}) =
    \begin{cases}
      (\infty, <)  & \mbox{if } m_{ij} > k \\
      (-k, <)  & \mbox{if } m_{ij} < -k \\
      m_{ij}, \smile _{ij} & \mbox{if } -k \leq m_{ij} \leq k
    \end{cases}
\end{displaymath}

\end{frame}

\section{Algorithms}

\begin{frame}{Creating the minimal zone graph}

\begin{itemize}

\item Origin of this algorithm - \cite{DBLP:conf/cav/GuhaNA12}

\item Minimal zone graph - stability, reachability, minimality.

\item Strategy: forward propagation for reachability and stability,
  backward propagation for stability.

\end{itemize}

\end{frame}

\begin{frame}{Creating the minimal zone graph - forward analysis}

\begin{itemize}

\item We use a queue, as in BFS, but we may visit a location multiple
  times unlike BFS, and unreachable locations may never be visited.

\item Each queue element stores a location
$l_{succ}$, its predecessor in the current path $l_{pred}$, and the
transition $t$ from $l_{pred}$ to $l_{succ}$. It

\item Each time a location
is visited, we create therein, new zones which are reachable from the
zones of the predecessor. \\
Thus, for each predecessor zone $(l_{pred}, \zeta _{pred_{i}})$, the
derived successor zone is $(l_{succ}, \zeta _{succ_{i}})$, where

\begin{displaymath} 
  \zeta _{succ_{i}} = ((\zeta _{pred_{i}} \uparrow \cap \texttt{t.guard})[\texttt{t.resets} := 0]) \uparrow
\end{displaymath} 

\end{itemize}

\end{frame}

\begin{frame}[shrink = 10]{Creating the minimal zone graph - forward analysis}

\begin{itemize}

\item Thus, if we let  $(l_{pred}, \zeta _{pred_{i}})$ range over the
existing zones of $l_{pred}$, and if we let  $(l_{succ}, \zeta
_{succ_{j}})$ range over the existing zones of $l_{succ}$, then the
new zones in $l_{succ}$ will cover

\begin{displaymath} 
  \zeta _{succ_{new}} = (\bigcup _{i} \zeta_{succ_{i}}) - (\bigcup _{j} \zeta_{succ_{j}})
\end{displaymath} 

\item Since $\zeta _{succ_{new}}$ is not necessarily convex, we may need to
split it into multiple convex polyhedra before creating the
corresponding zones in the $l_{succ}$. 

\item We may also need to split it further in order for the zones to
  be stable with respect to its outgoing edge guards, and also in
  order to filter out zones which do not satisfy its invariant.

\item We enqueue all the successors if any new zones are created this
  way, or if we are visiting the initial location for the first time.

\item We terminate when the queue is empty.

\end{itemize}

\end{frame}

\begin{frame}[shrink = 10]{Creating the minimal zone graph - backward analysis}

\begin{itemize}

\item We
iterate through the transitions of the timed automaton, multiple times
if necessary, splitting the zones of the source of each transition
with respect to the zones of the transition's target, until we achieve
stability of each zone of each location.

\item We recall that for stability,
whenever we have an edge in the zone valuation graph from a zone
$(l_{pred}, \zeta _{pred})$ to $(l_{succ}, \zeta _{succ})$
corresponding to a transition $t$ in the timed automaton, we require 
\begin{displaymath} 
  \zeta _{pred} \subseteq \texttt{t.guard}
  \wedge
  \zeta _{pred} [\texttt{t.resets} := 0] \subseteq \zeta _{succ}
\end{displaymath} 
Thus, when this does not hold, we split $(l_{pred}, \zeta _{pred})$
into
\begin{displaymath} 
  (l_{pred}, \zeta _{pred} \cap (\texttt{t.guard} \cap [\texttt{t.resets} := 0] \zeta _{succ}))
\end{displaymath} 
(which is convex and has an edge to $(l_{succ}, \zeta _{succ})$)
and
\begin{displaymath} 
  (l_{pred}, \zeta _{pred} - (\texttt{t.guard} \cap [\texttt{t.resets} := 0] \zeta _{succ}))
\end{displaymath} 
(which does not have an edge to $(l_{succ}, \zeta _{succ})$ and may
need to be split into convex zones.) \\
This generates the zone valuation graph.

\end{itemize}

\end{frame}

\begin{frame}[fragile, shrink=20]
  \frametitle{Creating the minimal zone graph - forward analysis}

  \begin{algorithm2e}[H]
    Initialise the queue $Q$ with a single element $(null, null, l_0)$\;
    Initialise the graph $zone\_graph$ with a single node $(l_0, v_0 \uparrow)$
    with an $\epsilon$ self-loop\;
    \While{$Q$ is not empty}{
      Dequeue $(l_{parent}, t, l_{child})$ from $Q$\;
      \If{$l_{parent} \neq null$}{
        \ForEach{zone $Z_{parent}$ of $l_{parent}$}{
          Add new zones to the zones of $l_{child}$ so that all zones
          reachable from $Z_{parent}$ are represented\;
          Abstract if necessary\;
          Update edges from $Z_{parent}$ to the new zones of $l_{child}$
          \If{new zones are created in $l_{child}$ or $l_{parent}$ is null}{
            \ForEach{outgoing transition $t'$ of $l_{child}$}{
              Enqueue $(l_{child}, t', \texttt{t'.target})$ in $Q$\;
            }
          }
        }
      }
    }
  \end{algorithm2e}
\end{frame}

\begin{frame}[fragile, shrink=20]
  \frametitle{Creating the minimal zone graph - backward analysis}

  \begin{algorithm2e}[H]
    Set $new\_zone$\;
    \While{$new\_zone$}{
      Reset $new\_zone$\;
      \ForEach{transition $t$ in the timed automaton}{
        Split the zones of \texttt{t.source} to be stable with respect to the
        zones of \texttt{t.target}\;
        Update edges accordingly\;
        \If{new zones are created in \texttt{t.source}}{
          Set $new\_zone$\;
        }
      }
    }
    Generate $minimal\_zone\_graph$ by applying Fernandez' algorithm to $zone\_graph$\;
    Return $zone\_graph$\;
  \end{algorithm2e}

\end{frame}

\begin{frame}{Minimal zone graph example}
  \begin{figure}
    \centering
    \def\svgwidth{0.6\columnwidth}
    \input{breaking2.pdf_tex}
    \caption{Timed automaton. Here, the states are \{0\}, the actions
      are \{a\}, and the clocks are \{X, Y\}.}
  \end{figure}
\end{frame}

\begin{frame}{Minimal zone graph example}
  \begin{figure}
    \centering
    \def\svgwidth{0.4\columnwidth}
    \input{breaking2-zones01.pdf_tex}
    \caption{Zones after one iteration.}
  \end{figure}
\end{frame}

\begin{frame}[shrink=20]{Minimal zone graph example}
  \begin{figure}
    \centering
    \def\svgwidth{0.7\columnwidth}
    \input{breaking2-zones02.pdf_tex}
    \caption{Zones after two iterations.}
  \end{figure}
\end{frame}

\begin{frame}[shrink=20]{Minimal zone graph example}
  \begin{figure}
    \centering
    \def\svgwidth{0.9\columnwidth}
    \input{breaking2-zones03.pdf_tex}
    \caption{Zones after three iterations without abstraction.}
  \end{figure}
\end{frame}

\begin{frame}[shrink=20]{Minimal zone graph example}
  \begin{figure}
    \centering
    \def\svgwidth{1.2\columnwidth}
    \input{breaking2-zones-abstracted.pdf_tex}
    \caption{Zones after three iterations with abstraction.}
  \end{figure}
\end{frame}

\begin{frame}[shrink=60]{Minimal zone graph example}
  \begin{figure}
    \centering
    \def\svgwidth{2.7\columnwidth}
    \input{breaking2-zones-quotient.pdf_tex}
    \caption{Zone graph with bisimilarity classes.}
  \end{figure}
\end{frame}

\begin{frame}{Verifying relations on pairs of timed automata.}
  \begin{itemize}
  \item Origin for this algorithm - \cite{arun2006bisimilarities} 
  \item General method to compute $(\rho, \sigma)$-bisimilarities on
    two LTS, starting from their initial locations.
  \item Can be adapted for a certain class of timed and time
    abstracted relations by using zone valuation graphs.
  \item For every relation $R$ satisfying this property, functions
    $f_P$ and $f_Q$ must exist such that the proposition $s_P R s_Q$
    resolves to one of these:
    \begin{itemize}
    \item yes
    \item no
    \item if and only if 
      \begin{align*} 
        &\forall (s_P', L_Q') \in f_P(s_P): \exists s_Q' \in
        L_Q': s_P' R s_Q' \quad \wedge \\
        &\forall (L_P', s_Q') \in f_Q(s_Q): \exists s_P' \in
        L_P': s_P' R s_Q'
      \end{align*} 
    \end{itemize}
  \end{itemize}
\end{frame}

\begin{frame}[shrink=20]{Verifying relations on pairs of timed automata.}
  \begin{itemize}
  \item For STaB, we define $f_P$ and $f_Q$ as
    \begin{align*}
      f_P(s_P) = & \{(s_P', L_Q') | s_P \xrightarrow{a} s_P', 
      L_Q=\{ s_Q' | s_Q \xrightarrow{a} s_Q'\}\} \\
      \cup & \{(s_P', L_Q') | s_P \xrightarrow{\epsilon} s_P', 
      L_Q=\{ s_Q' | s_Q \xrightarrow{\epsilon} s_Q'\}\} \\
      f_Q(s_Q) = & \{(L_P', s_Q') | s_Q \xrightarrow{a} s_Q', 
      L_P=\{ s_P' | s_P \xrightarrow{a} s_P'\}\} \\
      \cup & \{(L_P', s_Q') | s_Q \xrightarrow{\epsilon} s_Q', 
      L_P=\{ s_P' | s_P \xrightarrow{\epsilon} s_P'\}\} 
    \end{align*}

  \item For TadB, we define $f_P$ and $f_Q$ as
    \begin{align*}
      f_P(s_P) = & \{(s_P', L_Q') | s_P \xrightarrow{a} s_P', 
      L_Q=\{ s_Q' | s_Q \xrightarrow{\epsilon}\xrightarrow{a} s_Q'\}\} \\
      \cup & \{(s_P', L_Q') | s_P \xrightarrow{\epsilon} s_P', 
      L_Q=\{ s_Q' | s_Q \xrightarrow{\epsilon} s_Q'\}\} \\
      f_Q(s_Q) = & \{(L_P', s_Q') | s_Q \xrightarrow{a} s_Q', 
      L_P=\{ s_P' | s_P \xrightarrow{\epsilon}\xrightarrow{a} s_P'\}\} \\
      \cup & \{(L_P', s_Q') | s_Q \xrightarrow{\epsilon} s_Q', 
      L_P=\{ s_P' | s_P \xrightarrow{\epsilon} s_P'\}\} 
    \end{align*}

  \item For TaoB, we define $f_P$ and $f_Q$ as
    \begin{align*}
      f_P(s_P) = & \{(s_P', L_Q') | s_P \xrightarrow{a} s_P', 
      L_Q=\{ s_Q' | s_Q \xrightarrow{\epsilon}\xrightarrow{a}\xrightarrow{\epsilon} s_Q'\}\} \\
      \cup & \{(s_P', L_Q') | s_P \xrightarrow{\epsilon} s_P', 
      L_Q=\{ s_Q' | s_Q \xrightarrow{\epsilon} s_Q'\}\} \\
      f_Q(s_Q) = & \{(L_P', s_Q') | s_Q \xrightarrow{a} s_Q', 
      L_P=\{ s_P' | s_P \xrightarrow{\epsilon}\xrightarrow{a}\xrightarrow{\epsilon} s_P'\}\} \\
      \cup & \{(L_P', s_Q') | s_Q \xrightarrow{\epsilon} s_Q', 
      L_P=\{ s_P' | s_P \xrightarrow{\epsilon} s_P'\}\} 
    \end{align*}

  \end{itemize}
\end{frame}

\begin{frame}[fragile, shrink=60]
  \frametitle{Verifying relations on pairs of timed automata.}
\begin{procedure}[H]
  \caption{CheckStatesRelation($P$, $Q$, $s_{P}$, $s_{Q}$,
    $yes\_ table$, $no\_ table$)}
  \label{algorithm:checkstatesrelation}
  \SetKwFunction{lookup}{lookup}
  \SetKwFunction{insert}{insert}
  \SetKwFunction{remove}{remove}
  \Begin{
      \eIf{\lookup{$yes\_ table$, $s_{P}$, $s_{Q}$}}{
        \KwRet true\;
      }{
        \eIf{\lookup{$yes\_ table$, $s_{P}$, $s_{Q}$}}{
          \KwRet false\;
        }{
          \insert{$yes\_ table$, $s_{P}$, $s_{Q}$}\;
          Set $v_P$\;
          \ForEach{$(s_P', L_Q')$ in $f_P(s_P)$}{
            Reset $v_P$\;
            \ForEach{$s_Q'$ in $L_Q'$}{
              \If{\CheckStatesRelation{$P$, $Q$, $s_{P}'$, $s_{Q}'$,
                  $yes\_ table$, $no\_ table$}}{
                Set $v_P$\;
              }
            }
          }
          Set $v_Q$\;
          \ForEach{$(s_Q', L_P')$ in $f_Q(s_Q)$}{
            Reset $v_Q$\;
            \ForEach{$s_P'$ in $L_P'$}{
              \If{\CheckStatesRelation{$P$, $Q$, $s_{P}'$, $s_{Q}'$,
                  $yes\_ table$, $no\_ table$}}{
                Set $v_Q$\;
              }
            }
          }
          \eIf{$v_P$ $\wedge$ $v_Q$}{
            \KwRet true\;
          }{
            \remove{$yes\_ table$, $s_{P}$, $s_{Q}$}\;
            \insert{$yes\_ table$, $s_{P}$, $s_{Q}$}\;
            \KwRet false\;
          }
        }
      }
    }

\end{procedure}

\end{frame}

\begin{frame}[fragile]
  \frametitle{Verifying relations on pairs of timed automata.}

\begin{procedure}[H]
  \caption{CheckAutomataRelation($T_P$, $T_Q$)}
  \label{algorithm:checkautomatarelation}
  \SetKwFunction{CheckStatesRelation}{CheckStatesRelation}
  \Begin{
      Create zone valuation graphs $G_P$, $G_Q$ of $T_P$, $T_Q$\;
      Find the zone $s_P$ in $G_P$ which contains the initial
      state of $T_P$\;
      Find the zone $s_Q$ in $G_Q$ which contains the initial
      state of $T_Q$\;
      Initialise $yes\_ table$ and $no\_ table$ to empty tables\;
      \KwRet \CheckStatesRelation{$G_P$, $G_Q$, $s_P$, $s_Q$,
        $yes\_ table$, $no\_ table$}\;
    }
\end{procedure}

\end{frame}

\begin{frame}[shrink=20]{Verifying relations: example}

\begin{figure}%
\centering
\subfloat[First]{
    \def\svgwidth{0.6\columnwidth}
    \input{pair03first.pdf_tex}
}\qquad
\subfloat[Second]{
    \def\svgwidth{0.6\columnwidth}
    \input{pair03second.pdf_tex}
}\\
\caption{Timed automata.}
\label{pair03}
\end{figure}

\end{frame}

\begin{frame}[shrink=40]{Verifying relations: example}

\begin{figure}%
\centering
\subfloat[First]{
    \def\svgwidth{0.8\columnwidth}
    \input{pair03first-quotient.pdf_tex}
}\qquad
\subfloat[Second]{
    \def\svgwidth{0.8\columnwidth}
    \input{pair03second-quotient.pdf_tex}
}\\
\caption{Zone valuation graphs.}
\label{pair03-quotients}
\end{figure}

\end{frame}

\begin{frame}[allowframebreaks]
  \frametitle{References}
  \bibliographystyle{splncs}
  \bibliography{presentation}
\end{frame}

\end{document}
