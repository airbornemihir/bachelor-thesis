\chapter{Introduction}

The work described in this thesis builds on the work of
\cite{DBLP:conf/cav/GuhaNA12} in which Guha et al described an
algorithm to generate \emph{zone valuation graphs} for timed automata
and an algorithm to use such zone-valuation graphs to determine
\emph{timed performance prebisimilarity} on pairs of timed
automata. Our aim was to implement these algorithms in a generalised
manner in order to verify various other time abstracted relations,
such as time abstracted bisimulations \cite{tripakis2001analysis} and
time abstracted simulation equivalence. Towards
this end, we studied the literature about timed automata as well as
various existing tools for verifying these
equivalences (such as \texttt{minim}, described in
\cite{tripakis2001analysis}). Our implementation, in OCaml, implements
several of these relations and leaves some scope for implementing
others.

This document proceeds by developing the relevant theory for labelled transition
systems and strong bisimilarity, and then continues with timed
automata and time abstracted bisimilarity in analogous fashion. We 
then detail several methods to discretise the state space of timed
automata, including region graphs, zone graphs, and zone valuation
graphs, and use region graphs to show that a finite representation of
the state space of a timed automaton is always decidable. We mention
difference bound matrices (DBM), a widely used representation for the time
constraints (i. e. convex polyhedra) that are used to describe
zones. Then, we explain abstractions, which are transformations on DBM
that replace each DBM by a DBM which is its superset and
which is equivalent to it modulo strong time abstracted
bisimilarity. This helps us prevent state space explosion in our
exploration algorithms which build the state space of a timed
automaton. We then explain the common features of certain time
abstracted relations which allow us to verify them in a generalised
manner, and explain our implementation.
